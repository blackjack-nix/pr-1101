	\subsection{Présentation générale}
Dans le cadre des ateliers de fin d'année de E1, nous avons choisi le sujet "traitement de données médicales", groupe capteur. En effet, l'acquisition de données et leur traitement nous paraissait plus attirant.\\
Nous avons pour but de conceptualiser un outil d'aide au suivi médical e-santé, et d'implémenter une solution, avec un prototype opérationnel sur les variations et les pathologies du rythme cardiaque.\\
L'objectif de ce projet est d'aider le professionnel médical à détecter une bradycardie ou une tachycardie en fonction des résultats de la mesure clinique du pouls (battements, SPO2, température). 

	\subsection{Problématique}
Comment le traitement informatique de données médicales, constituées de mesures et de données cliniques, peuvent aider un professionnel de santé à prendre une décision la plus exacte possible ? 

	\subsection{Objectif}
L'objectif de ce projet était de récupérer les mesures cliniques enregistrées sur une plateforme forme e-santé via un web-service. Ces données étaient stockées dans une base de donnée. Il a fallu alors développer une application web qui traite les données récupérées et affiche les résultats en vue d'une aide au diagnostic par un professionnel de santé.
